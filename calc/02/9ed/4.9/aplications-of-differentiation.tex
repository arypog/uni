\documentclass[12pt]{article}
\usepackage[margin=1in]{geometry}
\usepackage[all]{xy}
\usepackage{mathtools}
\usepackage{relsize}
\usepackage{amsmath,amsthm,amssymb,color,latexsym}
\usepackage{geometry}        
\geometry{letterpaper}    
\usepackage{graphicx}
\usepackage{tikz}
\usetikzlibrary{arrows,positioning,shapes,fit,calc}
\pgfdeclarelayer{background}
\pgfsetlayers{background,main}

% Counters for statistics
\newcounter{problemcount}
\newcounter{subproblemcount}
\newcounter{solutioncount}

% Defining problem, subproblem and solution:
\newtheorem{problem}[problemcount]{Problem}
\newtheorem{subproblem}{\refstepcounter{subproblemcount}}[problem]
\makeatletter
\renewcommand{\thesubproblem}{%
  \theproblem-\ifnum\value{subproblem}<27%
    \alph{subproblem}%
  \else%
    \number\numexpr\value{subproblem}-26\relax%
  \fi%
}
\makeatother
\newenvironment{solution}[1][\it{Ans}]{
  \refstepcounter{solutioncount}{\textbf{#1.}\\}}
{\qedsymbol}

% Define the \concept command
\newcommand{\concept}[2]{%
  \begin{center}
  \fbox{\begin{minipage}{0.35\textwidth}
    \textbf{#1}.
    #2
  \end{minipage}}
  \end{center}
}

% Define the \rules command
\newcommand{\rules}[1]{%
  \begin{center}
  \fbox{\begin{minipage}{0.50\textwidth}
            #1
  \end{minipage}}
  \end{center}
}


\newcommand\problemnumbers{pg.332 15-26, 28, 35-47, 57, 67, 73, 75}


\begin{document}
% title
\noindent Cálculo - Capítulo 4.9.\hfill \problemnumbers\\
Ary Gomes da Costa. \hfill 15/02/2025\\

\textbf{5-26} Determine a primitiva mais geral da função. (Confira sua resposta derivando-a.)

\rules{
		$\int c \, dx = cx$.
}

\rules{
		$\int cf(x) \, dx = cF(x)$.
}

\rules{
		$\int f(x) + g(x) \, dx = F(x) + G(x)$.
}

\rules{
		$\int x^n \, dx = \frac{x^{n+1}}{n+1} \quad \text{(for } n \neq -1\text{)}$.
}

% % 15
\setcounter{problemcount}{15 - 1}
\begin{problem}
\[
    f(t) = \frac{2t - 4 + 3\sqrt{t}}{\sqrt{t}}
\]
\end{problem}

\begin{solution}
    \begin{equation}
        \begin{aligned}
            f(t)    & = 2t - 4 + 3 \times t^{\frac{1}{2}} \\
            F(t)    & = 2 \times \frac{t^2}{2} - 4t + 3 \times \frac{t^{\frac{1}{2}+1}}{\frac{1}{2} + 1} \\
                    & = t^2 - 4t + 3 \times \frac{t^{\frac{3}{2}} }{3 / 2} \\
                    & = t^2 - 4t + 3 \times 2 \times \frac{t^{3/2}}{3} \\
                    & = t^2 - 4t + 2t^{\frac{3}{2}} + C
        \end{aligned}
    \end{equation}
\end{solution}


\begin{problem}
\[
    f(x) = \sqrt[4]{5} + \sqrt[4]{x}
\]
\end{problem}

\begin{solution}
    \begin{equation}
        \begin{aligned}
            f(x) & = 5^{\frac{1}{4}} + x^{\frac{1}{4}} \\
            F(x) & = 5x^{\frac{1}{4} + 1} + \frac{x^{\frac{1}{4} + 1}}{\frac{1}{4} + 1} \\
                 & = 5x^{\frac{1}{4}} + \frac{x^{\frac{5}{4}}}{5/4} \\
                 & = 5x^{\frac{1}{4}} + \frac{4x^{\frac{5}{4}}}{5} + C
        \end{aligned}
    \end{equation}
\end{solution}

\rules{
		$\int \frac{1}{x} \, dx = \ln |x|  $.
}

\begin{problem}
\[
    f(x) = \frac{2}{5x} - \frac{3}{x^2}
\]
\end{problem}

\begin{solution}
    \begin{equation}
        \begin{aligned}
            f(x) & = \frac{2}{5} \times \frac{1}{x} + 3x^{-2} \\
            F(x) & = \frac{2}{5}x \times ln|x| + 3 \times x^{-2 + 1} \\
                 & = \frac{2}{5}ln|x| + \frac{3}{x}
        \end{aligned}
    \end{equation}
\end{solution}

\begin{problem}
\[
    f(x) = \frac{5x^2 - 6x + 4}{x^2} ,\ \ x > 0
\]
\end{problem}


\begin{solution}
    \begin{equation}
        \begin{aligned}
            f(x) & = (5x^2 - 6x + 4) \times x^{-2} \\
                 & = 5x^2 \times x^{-2} - 6x \times x^{-2} + 4 \times x^{-2} \\
                 & = 5 - 6x^{-1} + 4x^{-2} \\
                 & = 5 - 6 \times \frac{1}{x} + 4x^{-2} \\
            F(x) & = 5x - 6 \ln |x| + 4 \times \frac{x^{-2 + 1}}{-2 + 1} \\
                 & = 5x - 6 \ln |x| - \frac{4}{x} + C
        \end{aligned}
    \end{equation}
\end{solution}

\rules{
		$\int e^x \, dx = e^x $.
}

\begin{problem}
\[
    g(t) = 7e^t - e^3
\]
\end{problem}

\begin{solution}
    \begin{equation}
        \begin{aligned}
            G(t) & = 7e^t - et^3 + C
        \end{aligned}
    \end{equation}
\end{solution}

\begin{problem}
\[
    f(x) = \frac{10}{x^6} - 2e^x + 3
\]
\end{problem}

\begin{solution}
    \begin{equation}
        \begin{aligned}
            f(x) & = 10x^{-6} - 2e^x + 3 \\
            F(x) & = 10 \times \frac{x^{-6 + 1}}{-6+1} - 2e^x + 3x \\
                 & = -\frac{2}{x^{5}} - 2e^x + 3x + C
        \end{aligned}
    \end{equation}
\end{solution}

\rules{
		$\int \cos x \, dx = \sin x  $.
}

\rules{
		$\int \sin x \, dx = -\cos x  $.
}

\rules{
		$\int \sec ^2 x \, dx = \tan x  $.
}

\rules{
		$\int \sec x \tan x \, dx = \sec x  $.
}


\begin{problem}
\[
    f(\theta) = 2 \sin \theta - 3 \sec \theta \tan \theta
\]
\end{problem}

\begin{solution}
    \begin{equation}
        \begin{aligned}
            F(\theta) & = -2\sin \theta - 3 \sec \theta +C 
        \end{aligned}
    \end{equation}
\end{solution}

\begin{problem}
\[
    h(x) = \sec ^2 x + \pi \cos x
\]
\end{problem}

\begin{solution}
    \begin{equation}
        \begin{aligned}
            H(x) & =  \tan x + \pi \sin x + C
        \end{aligned}
    \end{equation}
\end{solution}

\rules{
		$\int \frac{1}{\sqrt{1 - x^2}} \, dx = \sin{^{-1}x}  $.
}

\rules{
		$\int \frac{1}{1 + x^2} \, dx = \tan{^{-1}x}  $.
}

\begin{problem}
\[
    f(r) = \frac{4}{1 + r^2} - \sqrt[5]{r^4}
\]
\end{problem}

\begin{solution}
    \begin{equation}
        \begin{aligned}
            f(r) & = 4 \times \frac{1}{1 + r^2} - r^{4/5} \\
            F(r) & = 4 \tan{^{-1}r} - \frac{r^{4/5 + 1}}{4/5 + 1} \\
                 & = 4 \tan{^{-1}r} - \frac{r^{9/5}}{9/5} \\
                 & = 4 \tan{^{-1}r} - \frac{5}{9}\sqrt[5]{r^9} + C
        \end{aligned}
    \end{equation}
\end{solution}

\begin{problem}
\[
    g(v) = 2\cos v - \frac{3}{\sqrt{1 - v^2}}
\]
\end{problem}

\begin{solution}
    \begin{equation}
        \begin{aligned}
            g(v) & = 2\cos v - 3 \times \frac{1}{\sqrt{1 - v^2}} \\
            G(v) & = 2 \sin v - 3 \sin{^{-1}v} + C
        \end{aligned}
    \end{equation}
\end{solution}

\rules{
		$\int b^x \, dx = \frac{b^x}{\ln b}$.
}

\rules{
		$\int \cosh x \, dx = \sinh x$.
}

\rules{
		$\int \sinh x \, dx =  \cosh x$.
}


\begin{problem}
\[
    f(x) = 2^x + 4 \sinh x
\]
\end{problem}

\begin{solution}
    \begin{equation}
        \begin{aligned}
            F(x) & = \frac{2^x}{\ln b} + 4 \cosh x + C \\
            
        \end{aligned}
    \end{equation}
\end{solution}

\rules{
		$\int \frac{1}{1 - x^2} \, dx = \tanh{^{-1}x}  $.
}

\rules{
		$\int \frac{1}{1 - x^2} \, dx = \frac{1}{2} \ln \left|\frac{x+1}{x-1} \right| $.
}

\rules{
		$\ln a - \ln b = \ln (\frac{a}{b})$.
}

\begin{problem}
\[
    f(x) = \frac{2x^2 + 5}{x^2 - 1}
\]
\end{problem}

\begin{solution}
    \begin{equation}
        \begin{aligned}
            f(x) & = \frac{2(x^2 - 1) + 7}{x^2 - 1} \\
                 & = 2 + \frac{7}{x^2 - 1} \\
            \frac{7}{x^2 + 1} & = \frac{A}{x-1} + \frac{B}{x+1}\\
            7 & = \frac{A(x + 1)(x - 1)}{(x -1)} + \frac{A(x + 1)(x - 1)} {(x + 1)} \\
            7 & = A(x+1) + B(x-1) \\
            7 & = Ax + A + Bx - B \\
            7 & = (A + B)x + (A - B) \\
            7 & = (0)x + (A - B) \\
            7 & = (A - B) \\
            B & = -A \\
            7 & = A -(-A) \\
            7 & = A + A \\
            A & = \frac{7}{2} \text{ \& } B = -\frac{7}{2}\\
            \frac{7}{x^2 + 1} & = \frac{7/2}{x-1} - \frac{7/2}{x+1}\\
              & = 2 + \frac{7}{2} \times \frac{1}{x - 1} - \frac{7}{2} \times \frac{1}{x + 1} \\
            F(x) & = 2x + \frac{7}{2} \ln |x-1| - \frac{7}{2} \ln |x+1| \\
                 & = 2x + \frac{7}{2} \left| \frac{\ln x - 1}{\ln x + 1} \right| + C
        \end{aligned}
    \end{equation}
\end{solution}

\textbf{27-28} Determine a função \textit{F} que seja primitiva de \textit{f} e satisfaça a
condição indicada. Confira sua resposta comparando os gráficos de
\textit{f} e \textit{F}.


\setcounter{problemcount}{28 - 1}

\begin{problem}
\[
    f(x) = 4 - 3(1 + x^2)^{-1} ,\ \ F(1) = 0
\]
\end{problem}

\begin{solution}
    \begin{equation}
        \begin{aligned}
            f(x) & = 4 - 3 \times \frac{1}{1 + x^2} \\
            F(x) & = 4x - 3 \tan{^{-1}x} + C\\
            F(0) & = 4 \times 0 - 3 \tan{^{-1}0} + C\\
            1    & = 0 - 0 + C \\
            C    & = 1 \\
            F(x) & = 4x - 3 \tan{^{-1}x} + 1\\
        \end{aligned}
    \end{equation}
\end{solution}

\textbf{29-54} Determine \textit{f}.

\setcounter{problemcount}{35 - 1}
\begin{problem}
\[
    f'''(t) = 12 + \sin t
\]
\end{problem}

\begin{solution}
    \begin{equation}
        \begin{aligned}
            F(t) = \ ?
        \end{aligned}
    \end{equation}
\end{solution}

% Statistics section
\section*\noindent
\textbf{N\textsuperscript{o} of Solutions:} \thesolutioncount{}
\hfill \problemnumbers


\end{document}
