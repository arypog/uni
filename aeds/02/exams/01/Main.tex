\documentclass{article}
\usepackage{amsmath, amssymb}
\usepackage{graphicx}
\usepackage{xcolor}
\usepackage{cancel}
\usepackage{tikz}

\title{Resumo: AEDS2 - prova 01}
\author{Ary}
\date{15/03/2025}

\begin{document}

\maketitle

\section{Algoritmo de Ordenação por Seleção}
A ordenação interna ocorre quando os elementos a serem ordenados cabem na memória principal.

O algoritmo segue os seguintes passos:
\begin{enumerate}
    \item Procura-se o menor elemento do array.
    \item Troca-se esse elemento com o primeiro elemento do array.
    \item Repete-se o processo para o restante do array até que a ordenação esteja completa.
\end{enumerate}

\subsection{Exemplo}
Dado o array inicial:
\begin{center}
    \texttt{[101, 115, 30, 63, 47, \textcolor{red}{20}]}
\end{center}
Após cada iteração:
\begin{center}
    \texttt{[20, 115, \textcolor{red}{30}, 63, 47, 101]}
\end{center}
\begin{center}
    \texttt{[20, 30, 115, 63, \textcolor{red}{47}, 101]}
\end{center}
\begin{center}
    \texttt{[20, 30, 47, \textcolor{red}{63}, 115, 101]}
\end{center}
\begin{center}
    \texttt{[20, 30, 47, 63, 115, \textcolor{red}{101}]}
\end{center}
\begin{center}
    \texttt{[20, 30, 47, 63, 101, 115]}
\end{center}

\subsection{Implementação em Pseudocódigo}
\begin{verbatim}
for (int i = 0; i < (n - 1); i++) {
    int menor = i; 
    for (int j = (i + 1); j < n; j++) {
        if (array[menor] > array[j]) {
            menor = j;
        }
    }
    swap(array[i], array[menor]);
}
\end{verbatim}

\subsection{Análise de Complexidade}
A complexidade do algoritmo de ordenação por seleção, em termos de comparações e movimentações, é:
\begin{itemize}
    \item \textbf{Melhor caso}: $O(n^2)$
    \item \textbf{Pior caso}: $O(n^2)$
    \item \textbf{Caso médio}: $O(n^2)$
\end{itemize}
O número de comparações pode ser expresso como a soma:
\begin{equation}
    \sum_{i=0}^{n-2} (n - 1 - i) = \frac{(n-1)n}{2} \Rightarrow O(n^2)
\end{equation}
Sendo que o loop interno repete $n - (i + 1)$ para $n-1$.\\
O número de trocas é limitado a $O(n)$.

\subsection{Conclusão}
O algoritmo de ordenação por seleção é simples e fácil de implementar, mas não é eficiente para grandes conjuntos de dados devido à sua complexidade quadrática.



\section{Somatórios}

\subsection{Definição}
\begin{equation}
    \sum_{i=a}^{b} f(i) = f(a) + f(a+1) + \dots + f(b)
\end{equation}


\subsection{Regras Básicas de Manipulação}
\begin{itemize}
    \item \textbf{Distributividade}: $\sum c f(i) = c \sum f(i)$
    \item \textbf{Associatividade}: $\sum (f(i) + g(i)) = \sum f(i) + \sum g(i)$
    \item \textbf{Comutatividade}: A ordem dos termos pode ser alterada sem mudar o resultado.
\end{itemize}

\subsection{Progressão Aritmética (PA)}
Uma PA é uma sequência cuja razão (diferença) entre dois termos
consecutivos é constante. Por exemplo, 5, 7, 9, 11, 13,

Cada termo da PA será a $i = a + b.i$, onde $a$ é o termo inicial; $b$, a razão; e, $i$, a ordem do termo.

Na sequência acima, $a$ e $b$ são iguais a 5 e 2, respectivamente. Logo, temos:
\begin{equation}
    (5 + 2.0), (5 + 2.1), (5 + 2.2), (5 + 2.3), (5 + 2.4), \,\, ...
\end{equation}

Aplique as regras de transformação para obter a fórmula fechada da soma $S_n$ dos
elementos de uma Progressão Aritmética (PA):

\begin{equation}
    \begin{aligned}
        s_n &= \sum_{i=0}^{n} (a + bi) \\
        \textit{Aplicando a Comutatividade}\\
            &= \sum_{i=0}^{n} [a + b(n-i)] = \sum_{i=0}^{n} [a + bn - bi)] \\ 
        2s_n&= \sum_{i=0}^{n} (a + bi) + \sum_{i=0}^{n} [a + bn - bi)] \\
        \textit{Aplicando a Associatividade}\\
            &= \sum_{i=0}^{n} [a + \cancel{bi} + a + bn - \cancel{bi})] \\
        2s_n&= \sum_{i=0}^{n} [2a + bn] \\
        \textit{Aplicando a Distributividade}\\
            &= (2a + bn) \sum_{i=0}^{n} 1 \\
        s_n &= \frac{(2a + bn) (n+1)}{2}
    \end{aligned}
\end{equation}

Fórmula fechada para o somatório de Gauss, usando a fórmula da soma de uma progressão aritmética qualquer:
\begin{equation}
    \begin{aligned}
        \sum_{i=0}^{n} i &= 0 + 1 + 2 + 3 + \dots + n \\   
        s_n &= \sum_{i=0}^{n} (0 + 1i) \quad  \overset{\text{a = $i_0 = 0$}}{\underset{\text{b = r = 1}}{\phantom{a}}}\\
            &= \frac{(\cancel{2\times0} + 1n)(n+1)}{2} \\
            &= \frac{n (n+1)}{2}
    \end{aligned}
\end{equation}



\subsection{Propriedades Somatorio}
\subsubsection{P1 - Combinando Conjuntos}
Combina conjuntos de índices diferentes. No caso, se $I$ e $I'$ são dois conjuntos
quaisquer de inteiros, então:
\begin{equation}
    \begin{aligned}
        \sum_{i \in I}a_i + \sum_{i \in I'}a_i = \sum_{i \in I \,\, I\cup I'}a_i + \sum_{i \in I \,\, I\cap I'}a_i \\
    \end{aligned}
\end{equation}
Se $A = {1, 2, 3}$ e $B = {3, 5, 7}$, então
$A \cup B = {1, 2, 3, 5, 7}$ e $A \cap B = {3}$
\subsubsection{P2 - Base para a Perturbação}
Dada uma soma genérica qualquer $\sum_{0\leq i\leq n} a_i$ temos que 
\begin{equation}
    \begin{aligned}
        s_{n+1} &= a_0 + a_1 + a_2 + \dots + a_n + a_{n+1} \\
        \textbf{1\textsuperscript{a} Forma} \\
        s_{n+1} &= \tikz[baseline]{\node[draw,circle,blue,inner sep=1pt] {$s_n + a_{n+1}$};} \\
        \textbf{2\textsuperscript{a} Forma} \\
        s_{n+1} &= \sum_{0\leq i \leq n+1} a_i 
            = a_0 +\sum_{1\leq i \leq n+1} a_i
            = \tikz[baseline]{\node[draw,circle,red,inner sep=1pt] {$a_0 + \sum_{1 \leq i \leq n} a_{i+1}$};} \\
        \textcolor{blue}{s_n + a_{n+1}} &= \textcolor{red}{a_0 + \sum_{1 \leq i \leq n} a_{i+1}}
    \end{aligned}
\end{equation}
Exemplo 1:
\begin{equation}
    \begin{aligned}
     s_n &= \sum_0^n ax^i \\
     s_{n+1} &= \textcolor{blue}{s_n + ax^{n+1}} =\textcolor{red}{ax^{0} +\sum_1^{n}ax^{i+1}} \\
    s_n + ax^{n+1} &= ax^{0} +\sum_1^{n}ax^{i+1} \\
    s_n + ax^{n+1} &= ax^{0} + a\sum_1^{n}x^{1}x^{i} \\
    s_n + ax^{n+1} &= ax^{0} + x^{1}a\sum_1^{n}x^{i}\\
    s_n + ax^{n+1} &= a\times1 + x\sum_1^{n}ax^{i}\\
    s_n + ax^{n+1} &= a + xs_n \\
    s_n + \cancel{ax^{n+1}} - xs_n \cancel{-ax^{n+1}} &= a + \cancel{xs_n} -ax^{n+1} - \cancel{xs_n} \\
    s_n - xs_n  &= a - ax^{n+1} \\
    s_n - xs_n  &= a - ax^{n+1} \\
    (1 - x)s_n  &= a - ax^{n+1} \\
    s_n &= \frac{a - ax^{n+1}}{1-x} \quad \textit{para } x \neq 1
    \end{aligned}
\end{equation}
Exemplo 2:
\begin{equation}
    \begin{aligned}
     s_n &= \sum_0^n i2^i \\
     s_n + (n+1)2^{n+1} &= 0 + \sum_0^n (i+1)2^{i+1} \\
     s_n + (n+1)2^{n+1} &= 0 + \sum_0^n i2^{i+1} + \sum_0^n 2^{i+1} \\
     s_n + (n+1)2^{n+1} &= 0 + 2\sum_0^n i2^{i} + \sum_0^n 2^{i+1} \\
     s_n + (n+1)2^{n+1} &= 2s_n + 2\sum_0^n 2^{i} \\
     s_n + (n+1)2^{n+1} &= 2s_n + 2\frac{1\times2^{n+1}}{-1} \\
     s_n + (n+1)2^{n+1} &= 2s_n + 2(2^{n+1}-1) \\
     (n+1)2^{n+1} - 2(2^{n+1}-1) &= 2s_n - s_n \\
     s_n &= (n+1)2^{n+1} - 2(2^{n+1}-1) \\
        &= n2^{n+1} - 2^{n+1} - 4^{n+1} + 2 \\
        &= n2^{n+1} - 2^{n+1} + 2 \\
    \sum_0^n i2^i &= (n - 1)2^{n+1} + 2
    \end{aligned}
\end{equation}

\subsection{Adivinhe a Resposta, Prove por Indução}
\subsubsection{1\textsuperscript{o} Passo (passo base)}
Provar que a fórmula é verdadeira para o primeiro
valor, substituindo n na equação pelo primeiro valor
\subsubsection{2\textsuperscript{o} Passo (indução propriamente dita)}
Supondo que $n > 0$ e que a fórmula é válida quando substituímos $n$ por $(n-1)$.

\[
    s_n = s_{n-1} + a_n}
\]
Pove por indução que a fórmula abaixo para a soma dos quadrados
perfeitos é verdadeira:
\begin{equation}
    \begin{aligned}
     \sum_0^n i^2 &= \frac{n(n+1)(2n+1)}{6} \quad \text{para }n \geq 0\\
     \textbf{1\textsuperscript{o} Passo} \\
     \sum_0^n i^2 &= \frac{0(0+1)(20+1)}{6}\\
     &= \frac{0}{6} = 0\\
     \textbf{2\textsuperscript{o} Passo} \\
     \frac{n(n+1)(2n+1)}{6} &= \frac{(n-1)((n-1)+1)(2(n-1)+1)}{6} + a_n\\
     \frac{(n^2 + n)(2n+1)}{6} &= \frac{(n-1)(n)(2n-1)}{6} + n^2 \\
    \frac{2n^3 + n^2 + 2n^2 + n}{6} &= \frac{(n^2 - n)(2n-1)}{6} + n^2\\
    \frac{2n^3 + 3n^2 + n}{6} &= \frac{2n^3 - n^2 - 2n^2 + n}{6} + n^2\\
    &= \frac{[2n^3 - n^2 - 2n^2 + n] + 6n^2}{6}\\
     \frac{2n^3 + 3n^2 + n}{6} &= \frac{2n^3 + 3n^2 + n}{6}\\
    \end{aligned}
\end{equation}
Encontre a fórmula fechada do somatório abaixo e, em seguida, prove a usando indução matemática:
\begin{equation}
    \begin{aligned}
     \sum_0^n 3 + i &= 3(n+1) + \frac{n(n+1)}{2} \\
                    &= \frac{6n+6 + n(n+1)}{2} \\ 
                    &= \frac{6n + 6 + n^2 + n}{2} \\
                    &= \frac{n^2 + 7n + 6}{2} \\
        \textbf{1\textsuperscript{o} Passo} \\
        \frac{0^2 + 7\times0 + 6}{2} &= 3 \\    
        \frac{0 + 0 + 6}{2} &= 3 \\
        3 &= 3 \\
        \textbf{2\textsuperscript{o} Passo} \\
        \frac{n^2 + 7n + 6}{2} &= \frac{(n-1)^2 + 7(n-1) + 6}{2} + a_n \\
        \frac{n^2 + 7n + 6}{2} &= \frac{(n^2 -2n +1) + (7n - 7) + 6}{2} + (3+n) \\
        &= \frac{(n^2 -2n +1) + (7n - 7) + 6 + (6 + 2n)}{2}\\
        \frac{n^2 + 7n + 6}{2} &= \frac{n^2 + 7n + 6}{2}\\
    \end{aligned}
\end{equation}

Encontre a fórmula fechada do somatório abaixo e, em seguida, prove a usando indução matemática:
\begin{equation}
    \begin{aligned}
     \sum_1^n [(2i + 1)^2 - (2i)^2] &= \sum_1^n [(4i^2 + 4i + 1) - (2i)^2] \\
     &= \sum_1^n [4i^2 + 4i + 1 - 4i^2] \\
     &= \sum_1^n [4i + 1] \\
     &= 4\sum_1^n i + \sum_1^n1 \\
     &= 4 \frac{n(n+1)}{2} + n \\
     &= 2n(n+1) + n \\
     &= 2n^2 + 3n \\
     \textbf{1\textsuperscript{o} Passo} \\
     2\times1^2 + 3\times1 &= [(2\times1 + 1)^2 - (2\times1)^2] \\
     2 + 3 &= [(3)^2 - 2^2] \\
     5 &= 9 - 4 \\
     5 &= 5 \\
     \textbf{2\textsuperscript{o} Passo} \\
     2n^2 + 3n &= 2(n-1)^2 + 3(n-1) + a_n \\
     2n^2 + 3n &= 2(n-1)^2 + 3(n-1) + [(2n + 1)^2 - (2n)^2] \\
            &= 2(n^2 - 2n + 1) + 3n - 3 + [4n^2 + 4n + 1 - 4n^2] \\
            &= 2n^2 - 4n + 2 + 3n - 3 + [4n + 1] \\
            &= 2n^2 + 3n \\
    \end{aligned}
\end{equation}
Perturbe o somatório dos cubos para encontrar a fórmula fechada do somatóriodos quadrados:
\begin{equation}
    \begin{aligned}
    s_n &= \sum_0^n i^2 \\
    scubo_n &= \sum_0^n i^3 \\
    scubo_{n} + (n+1)^3 &= 0^3 + \sum_0^n (i+1)^3 \\
    scubo_{n} + (n+1)^3 &= 0^3 + \sum_0^n [i^3 + 3i^2 + 3i + 1] \\
    scubo_{n} + (n+1)^3 &= \sum_0^n i^3 + \sum_0^n 3i^2 + \sum_0^n 3i + \sum_0^n 1 \\
   scubo_{n} + (n+1)^3 &= scubo_n + 3s_n + \frac{3n(n+1)}{2} + (n+1) \\
   (n+1)^3 &= 3s_n + \frac{3n(n+1)}{2} + (n+1) \\
   (n+1)^3 &= \frac{6s_n + 3n(n+1) + 2(n+1)}{2} \\
   2(n+1)^3 - 3n(n+1) - 2(n+1) &= 6s_n \\
   6s_n &= 2n^3 + 6n^2 + 6n + 2 - 3n^2 - 3n - 2n -2 \\
   6s_n &= 2n^3 + 3n^2 + n\\
   s_n &= \frac{2n^3 + 3n^2 + n}{6}\\
    \end{aligned}
\end{equation}

\subsubsection{Faça um método int somatorioPA(double a, double b, int n) que retorna o somatório dos n primeiros termos de uma PA com termo inicial a e razão b.}
\begin{verbatim}
int somatorioPA(double a, double b, int n) {
    return (2 * a + b * n) * (n + 1) / 2;
}
\end{verbatim}

\subsubsection{Um algoritmo de ordenação tradicional é o Inserção. Faça a análise decomplexidade desse algoritmo para os números de comparações e
movimentações entre registros no pior e melhor caso}
\begin{verbatim}
void insertion(int *arr, int n) {
    for (int i = 1; i < n; i++) {
        int key = arr[i];
        int j = i - 1;
        while (j >= 0 && arr[j] > key) {
            arr[j + 1] = arr[j];
            j = j - 1;
        }
        arr[j + 1] = key;
    }
}
\end{verbatim}
\begin{equation}
    \begin{aligned}
    \textit{Laço interno:}\\
    I(i-1) &= \sum_0^{i-1}1 = i \\
    \textit{Laço Externo:} \\
    T(n-1) &= \sum_0^{n-1}I(i-1) = \sum_0^{n-1}i = \frac{(n-1)(n)}{2}\\
    \textit{Melhor caso:} \\
    C(n) &= (n - 1) = \Theta(n) \\
    \textit{Pior caso:} \\
    C(n) &= \frac{(n-1)(n)}{2} = \Theta(n^2) \\
    Mi(n) &= C(n) - 1 \\
    M(n) &= Mi(n) + 2
    \end{aligned}
\end{equation}
https://www.cs.umd.edu/users/meesh/cmsc351/mount/lectures/lect3-sums-and-loops.pdf

\section{Estruturas de dados básicas lineares}
\subsection{Lista linear}
Tipo Abstrato de Dados (TAD) no qual podemos inserir e remover elementos em qualquer posição. Composta por um \textbf{Array} (de elementos) e um \textbf{n} (contador). A remoção é logica.
\begin{verbatim}
void inserirInicio(int x) {
    if (n >= MAXTAM)
        exit(1);
    //levar elementos para o fim do array
    for (int i = n; i > 0; i--){
        array[i] = array[i-1];
    }
    array[0] = x;
    n++;
}

void inserir(int x, int pos) {
    if (n >= MAXTAM || pos < 0 || pos > n)
        exit(1);
    //levar elementos para o fim do array
    for (int i = n; i > pos; i--){
        array[i] = array[i-1];
    }
    array[pos] = x;
    n++;
}

void inserirFim(int x) {
    if (n >= MAXTAM)
        exit(1);
    array[n] = x;
    n++;
}

int removerInicio() {
    if (n == 0)
        exit(1);
    int resp = array[0];
    n--;
    for (int i = 0; i < n; i++){
        array[i] = array[i+1];
    }
    return resp;
}

int remover(int pos) {
    if (n == 0 || pos < 0 || pos >= n)
        exit(1);
    int resp = array[pos];
    n--;
    for (int i = pos; i < n; i++){
        array[i] = array[i+1];
    }
    return resp;
}

int removerFim() {
    if (n == 0)
        exit(1);
    return array[--n];
}
\end{verbatim}
\subsection{Pilha (stack)}
Primeiro elemento que entra é o último a sair. Pilha de prato.

\begin{center}
    First In Last Out (FILO)
\end{center}

Tem basicamente os métodos de inserir (empilhar, push) e remover (desempilhar, pop).

\begin{verbatim}
int array[];
int n; 

void push(int x) {
    if (n >= MAXTAM)
        exit(1);
    array[n] = x;
    n++;
}

int pop() {
    if (n == 0)
        exit(1);
    return array[--n];
}
\end{verbatim}

\subsection{Fila (Queue)}
Are you a FIFO? Because you are a Queue\textless T\textgreater. haha.

Primeiro elemento que entra é o primeiro a sair. Tem basicamente os métodos de inserir (enfileirar, enqueue) e remover (desenfileirar, dequeue). Fila de banco.

As formas de implementar são: 
\begin{itemize}
    \item \textbf{Inserir Final e Remover Início (IF e RI) (FIFO)}: A remoção não é eficiente.
    \item \textbf{Inserir Início e Remover Final (II e RF) (LIFO)}: A 
\end{itemize}
Como implementar uma fila sem que uma das operações
desloque todos os elementos?

\textbf{R:} Fazendo uma fila circular, ou seja, depois da última posição,
retornamos para a primeira
\begin{verbatim}
class Fila {
    int[] array;
    int primeiro, ultimo;

    Fila (int tamanho) {
        array = new int[tamanho+1];
        primeiro = ultimo = 0;
    }

    void inserir(int x) throws Exception {
        if (((ultimo + 1) % array.length) == primeiro)
            throw new Exception("Erro!");
        array[ultimo] = x;
        ultimo = (ultimo + 1) % array.length;
    }

    int remover() throws Exception {
        if (primeiro == ultimo)
            throw new Exception("Erro!");
        int resp = array[primeiro];
        primeiro = (primeiro + 1) % array.length;
        return resp;
    }

    void mostrar() {
        int i = primeiro;
        System.out.print("[");
        while (i != ultimo) {
            System.out.print(array[i] + " ");
            i = (i + 1) % array.length;
        }
        System.out.println("]");
    }
\end{verbatim}

\end{document}

\end{document}
